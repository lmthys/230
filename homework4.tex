% some document settings
\documentclass[12pt]{article}
\setlength{\oddsidemargin}{12pt}
\setlength{\textwidth}{6.5in}
\setlength{\textheight}{9in}
\pagestyle{empty}
\setlength{\parskip}{7pt plus 2pt minus 2pt}

% include some needed libraries
\usepackage{amssymb}
\usepackage{amsmath}
\usepackage{amsthm}
\usepackage{enumitem}
\let\biconditional\leftrightarrow

\begin{document}
% begin editing

% title part
\title{
\textbf{COMS 230: Discrete Computational Structures}\\
Homework \# 4 \\}
\author{Larisa Andrews}
\date{\today}
\maketitle

\begin{enumerate}

% #1
\item {\bf Question 1} \\
Assume we have sets A,B,C \\
A = (a,b,c) , B = (d,e,f) , C = (g,h,i) \\
Then $B \cap C = ${} \\
Then $A \cup (B \cap C) = $(a,b,c) \\
Then $A \cup B = $(a,b,c,d,e,f) and $A \cup C = $(a,b,c,g,h,i)\\
Then $(A \cup B) \cap (A \cup C) = $(a,b,c)\\
Since (a,b,c) = (a,b,c) then $A \cup (B \cap C) = (A \cup B) \cap (A \cup C)$ \\
	
	
	
	

% #2
\item {\bf Question 2} \\
$A \cup U = U$ \\
$A \cup U$ can be expressed like $(x | (x \in A) \lor (x \in U))$ \\
then since we know $x \in U$, $ (x | (x \in A) \lor T)$ \\
then $(x | T) = U$\\
$A \cap \emptyset = \emptyset$ \\
$A \cap \emptyset$ can be expressed like $(x | (x \in A) \land (x \in \emptyset))$ \\
then since we know $x \notin \emptyset$, $ (x | (x \in A) \land F)$ \\
then $ (x | F) = \emptyset$ \\
% #3
\item {\bf Question 3} \\
a) Counter Example:\\
 let $ A = (2,3,4), B = (0,1,2), c = (0,1,3,4)$\\
 then $ A \cup C = (0,1,2,3,4)$ and $ B \cup C = (0,1,2,3,4)$ \\
 so $ A \cup C = B \cup C$ but $A \ne B $ \\
a) Counter Example:\\
 let $ A = (2,4,6,8), B = (0,2,4,6), c = (2,4,6)$\\
 then $ A \cap C = (2,4,6)$ and $ B \cap C = (2,4,6)$ \\
 so $ A \cap C = B \cap C$ but $A \ne B $ \\


% #4
\item {\bf Question 4} \\
Suppose $ A \cup C = B \cup C \land A \cap C = B \cap C \rightarrow A \ne B$ \\
let $ x \in A \land x \notin B \land x \in C$ \\
then $ x \in (A \cup C = B \cup C) $ so that $ x \in A \land x \notin B \land x \in C$ \\
then $ x \in (A \cap C = B \cap C) $so that $ x \in A \land x \in C \land x \notin B$ but by def of intersection \\
$ x \in B $ which is a contradiction. \\
So therefore  $ A \cup C = B \cup C \land A \cap C = B \cap C \rightarrow A = B$ \\
% #5 
\item {\bf{Question 5}} \\
$ S = A \cup (B \cap C), T = (A \cup B) \cap (A \cup C)$\\
$ S \subseteq T$ if $ x \in S \land x \in T$ \\
Lets assume $ x \in S \land x \notin T$ \\
then $ x \in S = A \cup (B \cap C)$ and $ x \notin T = (A \cup B) \cap (A \cup C)$ \\
Since by the assoc. law $ T = A \cup (B \cap C)$ then $x \in T$\\
which is a contradiction so $x \in S \land x \in T$ \\
So $ S \subseteq T$ \\

% #6
\item {\bf{Question 6}} \\
F is one to one because F(m,n) = F(x,y) \\
so $m+n = x+y , m-n = x-y$ \\
by adding the equations together we get \\
$ 2m = 2x \rightarrow m = x$ then substitute that in for m \\
$ (x)-n = x-y \rightarrow -n = -y \rightarrow n = y$ \\
so $ m = x, n = y$ \\
F is not onto because the inverse function \\
$(n-m, m+n)$ does not map to all real numbers. \\
  
% #7 
\item {\bf{Question 7}} \\
F is one to one because F(m,n) = F(x,y) \\
so $m+n = x+y$ \\
then if we assume $m+n$ is some real number $m$ and $x+y$ is a real number $x$ then $m = x$ \\

F is onto because the inverse function \\
$n-m = x$ where $x$ is any real number that the function maps to. \\
  

\end{enumerate}
\end{document}
