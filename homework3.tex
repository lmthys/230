% some document settings
\documentclass[12pt]{article}
\setlength{\oddsidemargin}{12pt}
\setlength{\textwidth}{6.5in}
\setlength{\textheight}{9in}
\pagestyle{empty}
\setlength{\parskip}{7pt plus 2pt minus 2pt}

% include some needed libraries
\usepackage{amssymb}
\usepackage{amsmath}
\usepackage{amsthm}
\usepackage{enumitem}
\let\biconditional\leftrightarrow

\begin{document}
% begin editing

% title part
\title{
\textbf{COMS 230: Discrete Computational Structures}\\
Homework \# 3 \\}
\author{Larisa Andrews}
\date{\today}
\maketitle

\begin{enumerate}

% #1
\item {\bf Question 1} \\
	Assume $p$ is an odd integer.\\
	$\Rrightarrow p = 2k+1$ for some $k  \epsilon  \mathbb{N}$ \\
	$\Rrightarrow p^3 = (2k+1)^3 = 8k^3  + 8k^2 + 4k +1 = 2(4k^3 + 4k^2 + 2k)+ 1$ \\
	$\Rrightarrow 4k^3 + 4k^2 + 2k$ is an integer.\\
	$\Rrightarrow p^3$ is an odd integer. \\
	
	
	

% #2
\item {\bf Question 2} \\
Since $x$ and $y$ are non-zero rational numbers, \\
let $ x = a/b $ and $y = c/d$ therefore $xy = ac/bd$ \\
Then if we divide this by $z$ we have $ac/bdz$ \\
Lets assume that the product of $ac/bdz$ is a rational number so, \\
$ac/bdz = e/f$ \\
$\Rrightarrow ac = ebdz/f \Rrightarrow acf = ebdz \Rrightarrow z = acf/ebd$\\
Because $acf/ebd$ is a rational number according to the definition. \\
This is a contradiction because $z$ is irrational. \\
Therefore $xy/z$ is irrational.\\
No, it is not direct because contradiction was used. \\
% #3
\item {\bf Question 3} \\
By contrapositive, if $n \ne$ even then, $5n+4 \ne even$ \\
$\Rrightarrow n = odd \rightarrow 5n+4 = odd$ \\
if $n$ is some odd integer $2k+1$ where $k \epsilon \mathbb{N}$\\
$\Rrightarrow 5(2k+1) + 4 = 10k +9 = 2(2k+4) + 1$ \\
$\Rrightarrow 2k+4$ is an integer plus 1 means it is odd. \\
$\Rrightarrow n = odd \rightarrow 5n+4 = odd$ \\
    


% #4
\item {\bf Question 4} \\
Lets assume that there are less than 5 meetings in one month. \\
That means that the max amount of meetings there could be in one month is 4. \\
Since $4* 12 = 48$ and there are 50 meetings this is a contradiction. \\
Therefore there has to be at least one month where there are 5 meetings. \\ 
% #5
\item {\bf{Question 5}} \\
case 1: $m > 0$ and $n > 0$ then $m * n = mn > 0$ \\
case 2: $m > 0$ and $n < 0$ then $m * n = mn < 0$ \\
case 3: $m < 0$ and $n > 0$ then $m * n = mn < 0$ \\
case 4: $m < 0$ and $n < 0$ then $m * n = mn > 0$ \\

% #6 
\item {\bf{Question 6}} \\
Suppose $\sqrt[3]{2}$ is rational. \\
Then there exists $p/q = \sqrt[3]{2}$ \\
$\Rrightarrow p^3/q^3 = 2 \Rrightarrow 2p^3 = q^3 \Rrightarrow p^3/2 = q^3$ \\
If $p^3$ can be divided by 2 then it has a factor of 2. \\
Let there be a $k$ where $k \epsilon \mathbb{N}$ and $p = 2k$ \\
$\Rrightarrow 2q^3 = (2k)^3 \Rrightarrow 2q^3 = 8k^3 \Rrightarrow q^3 = 4k^3$ \\
So $q^3$ has a factor of 2 which is a common factor of  $p^3$ \\
Contradiction: p and q cannot have common factors and be rational so, \\
Therefore $\sqrt[3]{2}$  is irrational.\\
% #7 
\item {\bf{Question 7}} \\
1.  $\sqrt{82} = 9.05$ \\
2.  $\sqrt{83} = 9.11$ \\
3.  $\sqrt{84} = 9.16$ \\
4.  $\sqrt{85} = 9.21$ \\
5.  $\sqrt{86} = 9.27$ \\
6.  $\sqrt{87} = 9.32$ \\
7.  $\sqrt{88} = 9.38$ \\
8.  $\sqrt{89} = 9.43$ \\
9.  $\sqrt{90} = 9.48$ \\
10. $\sqrt{91} = 9.53$ \\
11. $\sqrt{92} = 9.59$ \\
12. $\sqrt{93} = 9.64$ \\
13. $\sqrt{94} = 9.69$ \\
14. $\sqrt{95} = 9.74$ \\
15. $\sqrt{96} = 9.79$ \\
16. $\sqrt{97} = 9.84$ \\
17. $\sqrt{98} = 9.89$ \\
18. $\sqrt{89} = 9.94$ \\
My proof is constructive because I gave a concrete example instead of assuming. \\
% #8 
\item {\bf{Question 8}} \\
a) $A = [4+6n | n \epsilon \mathbb{N}]$ \\
b) $A = [n^2 -1 | n \epsilon \mathbb{N}]$ \\
% #9
\item {\bf{Question 9}} \\
For any $A = {x,y,z}, B = {a,b,c}, C = {d,e,f}$ \\
$(A x B) x C = {((x,a)d) ...}, A x (B x C) = { (x, (a,d)) ... }$ \\
$ ((x,a),d) \ne (x, (a,d))$ \\
Therefore $(A x B) x C \ne A x (B x C)$

\end{enumerate}
\end{document}
