% some document settings
\documentclass[12pt]{article}
\setlength{\oddsidemargin}{12pt}
\setlength{\textwidth}{6.5in}
\setlength{\textheight}{9in}
\pagestyle{empty}
\setlength{\parskip}{7pt plus 2pt minus 2pt}

% include some needed libraries
\usepackage{amssymb}
\usepackage{amsmath}
\usepackage{amsthm}
\usepackage{enumitem}
\let\biconditional\leftrightarrow

\begin{document}
% begin editing

% title part
\title{
\textbf{COMS 230: Discrete Computational Structures}\\
Homework \# 2 \\}
\author{Larisa Andrews}
\date{\today}
\maketitle

\begin{enumerate}

% #1
\item {\bf Question 1} \\
	universe: All students in class \\
	$S(x)$: x is a sophomore. \\
	$H(x)$: x has done their homework. \\
	a) $\forall xS(x)$ \\
	b) $\exists xH(x)$ \\
	universe: all ISU students. \\ 
	$C(x)$: x is a student in class \\
	a) $\forall x[C(x) \rightarrow S(x)]$ \\
	b) $\exists x[C(x) \land H(x)]$ \\
	

% #2
\item {\bf Question 2} \\
Let $P(x)$ be "x is true when lights are off" \\
and let $Q(x)$ be "x is true when lights are on" \\
There will never be a world where the lights are on and off. \\
But there may be a world where the lights are on and another world where the lights are off. \\
Therefore $\forall x(P(x) \biconditional Q(x)) \not\equiv$ \\
$\forall xP(x) \biconditional  \forall xQ(x)$ 


% #3
\item {\bf Question 3} \\
a) $\exists x \exists y \forall z[(x \ne y \land S(x) \land S(y)) \land (F(z) \rightarrow (A(z,y) \lor A(z,x)))]$ \\
b) $\forall x \exists y \exists z[F(x) \rightarrow ((S(y) \land A(x,y)) \land (S(z) \land A(x,z)))]$\\


% #4
\item {\bf Question 4} \\
Assuming $\exists x \land \exists y \land y \ne x$ \\
a) $ x = F(x,y) \lor F(y,x) $ \\
b) $ P1 = \forall y(F(x,y)) $ \\
c) $ P = F(y,x)$ \\
d) $ P3 = F(P1, x)$\\

% #5
\item {\bf{Question 5}} \\
a) universe: a student in class \\
$L(x)$: x has visited London \\
$B(x)$: x has visited bucking ham palace \\
\begin{align}
\textrm{Left column} && \textrm{Right column} \\
      \exists xL(x) && \textrm{Hyp 1} \\
  	  \forall x[L(x) \rightarrow B(x)] && \textrm{Hyp 2} \\
      L(j) \rightarrow B(j) \textrm{(for any j)} && \textrm{Universal Instantiation} \\
      L(j) \textrm{(for any j)} && \textrm{Universal Instantiation} \\
      B(j) &&  \textrm{Modus Ponens (4), (5)} \\
      \exists xB(x) && \textrm{Existential Generalization} 
\end{align}

b)
universe: all movies \\
$P(x)$: x is a popular movie. \\ 
$I(x)$: x is a movie made in Iowa. \\
$A(x)$: x is an action movie. \\
\begin{align}
\textrm{Left column} && \textrm{Right column} \\
      \forall x[A(x) \rightarrow P(x)] && \textrm{Hyp 1} \\
  	  \exists x [A(x) \land I(x)] && \textrm{Hyp 2} \\
      A(b) \rightarrow P(b) \textrm{(for all b)} && \textrm{Universal Instantiation} \\
      A(b) \land I(b) \textrm{(for all b)} && \textrm{Universal Instantiation} \\
      A(b) \textrm{(for all b)} &&  \textrm{Simplification(12)} \\
      P(b) \textrm{(for all b)} && \textrm{Modus ponens, (13) , (11)} \\ 
      I(b) \textrm{(for all b)} && \textrm{Simplification (12)} \\
      P(b) \land I(b) \textrm{(for all b)} && \textrm{Conjuction(14) , (15)} \\
      \exists x[P(x) \land I(x)] && \textrm{Existential Generalization} 
\end{align}


% #6 
\item {\bf{Question 6}} \\
a) Universe: All People \\
$H(x)$: x lives in Hawaii \\
$O(x)$: x lives close to the ocean \\
$S(x)$: x knows how to surf \\
$B(x)$: x owns a surf board \\
\begin{align}
\textrm{Left column} && \textrm{Right column} \\
      \forall x[H(x) \rightarrow O(x)] && \textrm{Hyp 1} \\
  	  \forall x [\lnot S(x) \rightarrow \lnot O(x)] && \textrm{Hyp 2} \\
      \forall x [S(x) \rightarrow B(x)] && \textrm{Hyp 3} \\
      H(a) \rightarrow O(a) \textrm{(for all a)} && \textrm{Universal Instantiation} \\
      \lnot S(a) \rightarrow \lnot O(a) \textrm{(for all a)} && \textrm{Universal Instantiation} \\
      S(a) \rightarrow B(a) \textrm{(for all a)} && \textrm{Universal Instantiation} \\
      O(b) \rightarrow S(a) \textrm{(for all a)} &&  \textrm{Contrapositive (23)} \\
      H(a) = True \textrm{(for all a)} && \textrm{Assumption that H(a) is true} \\ 
      O(a) \textrm{(for all a)} && \textrm{Modus Ponens,(22), (26)} \\
      S(a) \textrm{(for all a)} && \textrm{Modus Ponens,(25), (27)} \\
      B(a) \textrm{(for all a)} && \textrm{Modus Ponens,(24), (28)} \\
      H(a) \rightarrow B(a) \textrm{(for all a)}  && \textrm{26} \\
      \forall x[H(x) \rightarrow B(x)] && \textrm{Existential Generalization} 
\end{align} \\
b) 
\begin{align}
\textrm{Left column} && \textrm{Right column} \\
	  \forall x[P(x) \rightarrow Q(x)] && \textrm{Hyp 1} \\
  	  \forall x [Q(x) \rightarrow R(x)] && \textrm{Hyp 2} \\
      P(a) \rightarrow Q(a) \textrm{(for all a)} && \textrm{Universal Instantiation} \\
      Q(a) \rightarrow R(a) \textrm{(for all a)} && \textrm{Universal Instantiation} \\
      P(a) = True \textrm{(for all a)} && \textrm{Assumption that P(a) is true} \\ 
      Q(a) \textrm{(for all a)} && \textrm{Modus Ponens,(35), (37)} \\
      R(a) \textrm{(for all a)} && \textrm{Modus Ponens,(36), (28)} \\
      P(a) \rightarrow Q(a) \textrm{(for all a)}  && \textrm{37} \\
      \forall x[P(x) \rightarrow Q(x)] && \textrm{Existential Generalization} 
\end{align} \\
c)	\\
\begin{align}
\textrm{Left column} && \textrm{Right column} \\
      \forall x[H(x) \rightarrow O(x)] && \textrm{Hyp 1} \\
  	  \forall x [\lnot S(x) \rightarrow \lnot O(x)] && \textrm{Hyp 2} \\
      \forall x [S(x) \rightarrow B(x)] && \textrm{Hyp 3} \\
      H(a) \rightarrow O(a) \textrm{(for all a)} && \textrm{Universal Instantiation} \\
      \lnot S(a) \rightarrow \lnot O(a) \textrm{(for all a)} && \textrm{Universal Instantiation} \\
      S(a) \rightarrow B(a) \textrm{(for all a)} && \textrm{Universal Instantiation} \\
      O(a) \rightarrow S(a) \textrm{(for all a)} &&  \textrm{Contrapositive (23)} \\
      H(a) \rightarrow S(a) \textrm{(for all a)} && \textrm{universal transitivity,(46),(49)} \\ 
      H(a) \rightarrow B(a) \textrm{(for all a)} && \textrm{universal transitivity,(48),(50)} \\
      \forall x[H(x) \rightarrow B(x)] && \textrm{Existential Generalization} 
\end{align} \\
Yes, my proof is shorter. 



\end{enumerate}
\end{document}
