% some document settings
\documentclass[12pt]{article}
\setlength{\oddsidemargin}{12pt}
\setlength{\textwidth}{6.5in}
\setlength{\textheight}{9in}
\pagestyle{empty}
\setlength{\parskip}{7pt plus 2pt minus 2pt}

% include some needed libraries
\usepackage{amssymb}
\usepackage{amsmath}
\usepackage{amsthm}
\usepackage{enumitem}
\let\biconditional\leftrightarrow

\begin{document}
% begin editing

% title part
\title{
\textbf{COMS 230: Discrete Computational Structures}\\
Homework \# 1 \\}
\author{Larisa Andrews}
\date{\today}
\maketitle

\begin{enumerate}

% #1
\item {\bf Question 1} \\
	a) It is not sunny and it is not snowing \\
	b) If it is not sunny and it is snowing, the 		John will go skiing. \\
	c) It is sunny and not snowing or it is snowing 	and John goes skiing. \\ 

% #2
\item {\bf Question 2} \\
a) 
   $p$ = "Passing the final",
   $q$ = "attending class regularly",
   $r$ = "to pass the class" \\
   in logical operations: $(p \lor q) \rightarrow r$ \\
b) 
   $p$ = "to pass the class",
   $q$ = "to attend class regularly" \\
   in logical operations: $ p \rightarrow q $ \\
c)
   $p$ = "you will pass the class",
   $q$ = "you attend class regularly",
   $r$ = "you pass the final" \\
   in logical operations: $ p \biconditional (q \wedge r) $


% #3
\item {\bf Question 3} \\
First, $a$ = Alice, $b$ = Bob, $c$ = Cindy, $d$ = Don, and $e$ = Ellen. Also, $vg$ = video games and $bb$ = basketball.

Lets assume $b$ plays $vg$. If $b$ plays $vg$ then so do $a$ and $e$ according to sentence 5. If $a$ plays $vg$ then so does $d$ according to sentence 4. If $e$ plays $vg$ then so does $c$ according to sentence 3. But according to sentence 2 $c$ and $d$ cannot play the same thing. So therefore $b$ does not play $vg$. 

So $b$ plays $bb$. So then $a$ plays $vg$ according to sentence 1. According to sentence 4 $d$ will play $vg$ as well. Then according to sentence 2 $c$ plays $bb$. Then due to the contrapositive of sentence 3 $e$ will play $bb$. 


% #4
\item {\bf Question 4} Demonstration of truth table


Demonstration of truth table $(\lnot p \land (p \rightarrow q)) \rightarrow \lnot q)$\\

\begin{tabular}{c|c|ccccc}
$p$ & $q$ & $(\lnot p $ & $\land$ & $(p \rightarrow q))$ & $\rightarrow$ & $\lnot q$\\ \hline
T & T & F & F & T & \textbf{T} & F\\
T & F & F & F & F & \textbf{T} & T\\
F & T & T & T & T & \textbf{F} & F\\
F & F & T & T & T & \textbf{T} & T\\
\end{tabular}\\\\
Because the boldface row is not always true the logical statement \\
$(\lnot p \land (p \rightarrow q)) \rightarrow \lnot q)$ is not a tautology. 
% #5 
\item {\bf{Question 5}}

\begin{tabular}{c|c|c|ccc}
$p$ & $q$ & $r$ & $(p $ & $\rightarrow$ & $(q \rightarrow r))$\\ \hline
T & T & T & T & \textbf{T} & T\\
T & T & F & T & \textbf{F} & F\\
T & F & T & T & \textbf{T} & T\\
T & F & F & T & \textbf{T} & T\\
F & T & T & F & \textbf{T} & T\\
F & T & F & F & \textbf{T} & F\\
F & F & T & F & \textbf{T} & T\\
F & F & F & F & \textbf{T} & T\\
\end{tabular}\\\\

% #6
\item {\bf{Question 6}}

a) Truth table for \\
\begin{tabular}{c|c|c|ccc|ccc}
$p$ & $q$ & $r$ & $(p \rightarrow r)$ & $\land$ & $(q \rightarrow r)$ & $(p \lor q)$ & $\rightarrow$ & $r$\\ \hline
T & T & T & T & T & T & T & T & T\\
T & T & F & F & F & F & T & F & F\\
T & F & T & T & T & T & T & T & T\\
T & F & F & F & F & T & T & F & F\\
F & T & T & T & T & T & T & T & T\\
F & T & F & T & F & F & T & F & F\\
F & F & T & T & T & T & F & T & T\\
F & F & F & T & T & T & F & T & F\\
\end{tabular}\\\\
b)
\begin{align}
\textrm{Left column} && \textrm{Right column} \\
     (p \rightarrow r) \land (q \rightarrow r) = (\lnot p \lor r) \land (\lnot q \lor r)  && \textrm{Table 7.1} \\
  	  = (\lnot p \lor \lnot q) \lor (r \lor r) && \textrm{Comm. and Assoc.} \\
      = (\lnot p \lor \lnot q) \lor r && \textrm{idempotent} \\
      = \lnot (p \land q) \lor r && \textrm{de Morgans} \\
      = p \land q \rightarrow r &&  \textrm{Table 7.1} 
\end{align}


% #7 
\item {\bf{Question 7}}


\begin{tabular}{c|c|ccc|cccc}
$p$ & $q$ & $p $ & $NOR$ & $q$ & $\lnot$ & $(p$ & $ \lor $ & $q)$\\ \hline
T & T & T & F & T & F & T & T & T\\
T & F & T & F & F & F & T & T & F\\
F & T & F & F & T & F & F & T & T\\
F & F & F & T & F & T & F & F & F\\
\end{tabular}\\\\
To prove NOR is functionally complete we must show how NOR can create a not, or, and and gate.\\
$p NOR q = \lnot (p \lor q)$  as proven by the above truth table\\
$\lnot p = p NOR p = \lnot (p \lor p)$ (1)  by idempotent\\
$ p \lor q = \lnot ( \lnot (p \lor q)) = (1) (p NOR q)$ (2) by double negation\\
$ p \land q = \lnot (\lnot p \lor \lnot q) = (1)( (1) p (2) (1) q )$ by de Morgans \\


\end{enumerate}
\end{document}
