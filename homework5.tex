% some document settings
\documentclass[12pt]{article}
\setlength{\oddsidemargin}{12pt}
\setlength{\textwidth}{6.5in}
\setlength{\textheight}{9in}
\pagestyle{empty}
\setlength{\parskip}{7pt plus 2pt minus 2pt}

% include some needed libraries
\usepackage{amssymb}
\usepackage{amsmath}
\usepackage{amsthm}
\usepackage{enumitem}
\let\biconditional\leftrightarrow

\begin{document}
% begin editing

% title part
\title{
\textbf{COMS 230: Discrete Computational Structures}\\
Homework \# 6 \\}
\author{Larisa Andrews}
\date{\today}
\maketitle

\begin{enumerate}

% #1
\item {\bf Question 1} \\
a) \\
Because f is a total function that means there is a mapping from every element in $ B \rightarrow C$ so it is also a one to one function.\\
B) \\
if $A = \{a\} , B = \{b,c\}, C = \{d,e,f\}$\\
and $(f \circ g)(d) = a \land (e) = a \land (f) = a$\\
and $ g(a) = b $ and $f(b) = d, f(c) = e $ \\
So f is not an onto function. \\ 

	
	
	
	

% #2
\item {\bf Question 2} \\
a) \\
This relation is reflexive when x is a real number greater than zero and anti-reflexive when  less than zero, because \\
when $x \geq 0$ then $(x,x) \in R1 \biconditional 2x \geq 0$ \\
else when $x < 0$ then $(x,x) \in R1 \ne 2x \geq 0$ \\
This relation is transitive because \\
$ (x,y) \in R1 \land (y,z) \in R1 \rightarrow (x,z) \in R1$ \\
$ xy \geq 0 \land yz \geq 0$ which means x,y, and z are all positive real numbers or 0 so, $yz \geq 0$ for all Real Numbers \\
This relation is also symmetric because \\
if $ xy \geq 0 $ then because of multiplication rules $ yx \geq 0 $ \\
b) \\
This relation is anti-reflexive because \\
$ (x,x) \notin R2 \rightarrow x \ne 2x$ \\
This relation is not transitive because \\
$ (x,y) \in R2 \land (y,z) \in R2$ does not imply $(x,y) \in R2$\\
$ x = 2y \land y = 2z \rightarrow x = 4z \ne x = 2z $ \\
This relation is anti-symmetric because \\
$ x = 2y \ne y = 2x $ and $ x = 2y \land y = 2x $ only implies $ y = x $ \\
when $ y = 0 \land x = 0 $ so for all other cases it is not true so because of the way an implies work the statement is true. \\ 


% #3
\item {\bf Question 3} \\
a)\\
R3 is reflexive because \\
$ (x,x) \in R3 \rightarrow x/x = 1, \forall x \in \mathbb{R} +$ and $ 1 \in \mathbb{Z} $ \\
R3 is anti- symmetric because \\
$ (x,y) \in R3 \land (y,x) \in R3 \rightarrow y = x $ \\
$ x/y \in \mathbb{Z} \land y/x \in \mathbb{Z}$\\ 
Since $ x/y = \mathbb{Z} => x = y\mathbb{Z} => 1/(\mathbb{Z}) = y/x$ \\
then $ 1/(\mathbb{Z}) \in \mathbb{Z} $ if $ x = y$ \\
R3 is transitive because \\
$ x/y \in \mathbb{Z} \land y/z \in \mathbb{Z}$ then $y = x/(\mathbb{Z})$\\
$ (x/(\mathbb{Z})) / z = \mathbb{Z} => x/(\mathbb{Z})(z) = \mathbb{Z}$ \\
$ x/z = 2(\mathbb{Z})$ since 2 is an even number $x/z \in \mathbb{Z}$\\
So R3 is a strict partial order. \\
b) \\
R4 is reflexive because \\
$ (x,x) \in R4 \biconditional x - x \in \mathbb{Z} $ and $ x - x = 0 $ for all $\mathbb{R}$ and $0 \in \mathbb{Z}$ \\
R4 is symmetric because \\
$ (x,y) \in R4 \rightarrow (y,x) \in R4 $ because if $ x - y \in \mathbb{Z} $ then $ y - x $ is the same number with the opposite sign but still an integer. \\
R4 is transitive because \\
$ (x - y) \in \mathbb{Z} \land (y - z) \in \mathbb{Z} = (x - (z+ \mathbb{Z}) = \mathbb{Z} = x - z = 0$ \\
$ 0 \in \mathbb{Z}$ \\ 
So R4 is an equivalence \\
R4 equivalence class for $\pi$ = [$\pi - 1$, $\pi - 2$, $\pi - 3$, ... ] \\
R4 equivalence class for $2$ = [$1$, $0$, $-1$, ...] \\



% #4
\item {\bf Question 4} \\
A) \\
R5 is reflexive because \\
$((a,a), (a,a) ) \in R5$ => $a/a = a/a$ 
R5 is symmetric because \\
$((a,b) , (e,d)) \in R5 \rightarrow((b,a),(d,e)))  $\\
$=> a/b = e/d$ \\
$=>  ad = be $ \\
$=>  d/e = b/a $ \\
$=>$ because '=' is symmetric $ b/a = d/e $ \\
R5 is transitive because \\
$((a,b), (c,d)) \land ((b,e), (d,f)) \in R5 \rightarrow ((a,e),(c,f)) \in R5$\\
$ a/b = c/d \land b/e = d/f $ \\
$=>  b = de/f $ \\
$=>  a/(de/f) = c/d $ \\
$=>  af/de = c/d $ \\
$=>  af(d) = c(de) $ \\
$=>  c = af/e $ \\
$=>  ce = af $ \\
$=>  c/f = a/e $ \\
$=>$ because '=' is symmetric $ a/e = c/f $ \\
B) \\
$f(a,b) = a/b$ \\
C)\\
(1,1) = [(1,1) , (2,2) , (3,3), ...$(\mathbb{Z}+ / \mathbb{Z}+)$] \\
D) \\
There is an equivalence class for every positive integer $n$, that is resulting from $a/b$ then for every $n$, there is a $c/d$ that equals $a/b$ \\
So for example for when $a/b = 1$ then $n = 1$ and the equivalence class contains every $c/d = 1$ \\


% #5 
\item {\bf{Question 5}} \\
reflexive: \\
$(f,f) \rightarrow F(0) = f(0) \land f(1) = f(1) $ \\
symmetric: \\
$ (f,g) \rightarrow (g,f) $ \\
$ f(0) = g(0) \land f(1) = g(1) $ because '=' is symmetric \\
$ g(0) = f(0) \land g(1) = f(1) $ \\
transitive: \\
if $ (f(0) = g(0) \land f(1) = g(1)) \land (g(0) = h(0) \land g(1) = h(1) $ \\
then because '=' is transitive $f(0) = h(0) \land f(1) = h(1) $ \\
so this relation is a equivalence relation. \\
where $f(n) = n$ then the equivalence class is  $[f] = [0,1]$ \\   


\end{enumerate}
\end{document}
