% some document settings
\documentclass[12pt]{article}
\setlength{\oddsidemargin}{12pt}
\setlength{\textwidth}{6.5in}
\setlength{\textheight}{9in}
\pagestyle{empty}
\setlength{\parskip}{7pt plus 2pt minus 2pt}

% include some needed libraries
\usepackage{amssymb}
\usepackage{amsmath}
\usepackage{amsthm}
\usepackage{enumitem}
\let\biconditional\leftrightarrow

\begin{document}
% begin editing

% title part
\title{
\textbf{COMS 230: Discrete Computational Structures}\\
Homework \# 7 \\}
\author{Larisa Andrews}
\date{\today}
\maketitle

\begin{enumerate}

% #1
\item {\bf Question 1} \\
a) 
	Base Case: f(1) = $1/(1*2) = 1/2 = 1/(1+1)$ so the base case holds. \\
	IH: Assume f(k) = $1/(1*2) + 1/(2*3) + ... + 1/k(k+1) = k/(k+1)$ \\
	Prove f(k+1) = the sum of $1/(1*2) + 1/(2*3) + ... + 1/k(k+1) + 1/k(k+2)$\\
	 since we assumed the sum of the first part is f(k) then we know \\
	 f(k+1) = $f(k) + 1/k(k+2) = k/(k+1) + 1/k(k+2)$ \\
	 $ = k*k(k+2)/k(k+2)(k+1) + (k+1)/k(k+2)(k+1)  = k^3+3k^2+k+1/k(k+2)(k+1)$ \\
	 $ =  (k+1)/(k+2) $ \\
	 So f(k+1) holds so therefor it holds for all positive integers n. 
	
	
	

b) \\
   Base Case: f(1) = $2-2*7^1 = 2-14 = -12 = (1-(-7^2))/4 = -12$ So the base case holds. \\
   IH: Assume f(k) = the sum of $2-2*7 + 2*7^2 - ... + 2(-7)^k = (1-(-7)^{k+1}))/4$ \\
   Prove: f(k+1) = the sum - $2(-7)^{k+1}$ \\
   so f(k) - $2(-7)^{k+1} = (1-(-7)^{k+1})/4 - 2(-7)^{k+1}$ \\
   $ = (1-(-7)^{k+1})/4 - 4(2*7^{k+1})/4 = (1-(-7)^{k+1} - 8*7^{k+1})/4$ \\ $ = (1-(-7)^{k+2})/4) $ \\
   So f(k+1) holds for all non-negative numbers k. \\
   
   
c) \\
	Base Case: f(1) = $1*2*3 = 6 = 1(1+1)(1+2)(1+3)/4 = 1*2*3*4/4 = 24/6 = 6 $ \\
	So the base case holds.\\
	IH: Assume that f(k) = the sum of $1*2*3 + 2*3*4 + ... + k(k+1)(k+2) = k(k+1)(k+2)(k+3)/4$ \\
	Prove for f(k+1) = the sum + $(k+1)(k+2)(k+3)$ so \\
	$ = f(k) + (k+1)(k+2)(k+3) = k(k+1)(k+2)(k+3)/4 + (k+1)(k+2)(k+3) $ \\
	$ k(k+1)(k+2)(k+3) + 4(k+1)(k+2)(k+3) / 4 = k^4 + 10k^3 + 35k^2 + 50k + 6 = (k+1)(k+2)(k+3)(k+4)/4 $\\
	So f(k+1) holds for all positive integers. \\
	
	
d) \\
	base Case: f(1) = $1^3 + 2(1) / 3 = 1+2/3 = 1$ which is an int so the base case holds \\
    IH: Assume f(k) = $k^3 + 2k /3 $ is an int. 
    Prove f(k+1) is an int as well. Then we know if f(k) is and int then 
    f(k) + k+1 would be an int. 
    then $k^3 + 2k /3 + k+1 = k^3 + 5k + 3/3 = (k+1)^3 + 2(k+1) - 3k^2/3 = (k+1)^3 + 2(k+1)/3 - 3k^2/3 $ \\Since $3k^2/3$ is divisible by three than for f(k+1) holds for all positive ints. \\
  
   

% #2
\item {\bf Question 2} \\
Base Case: one unit square is 4 and f(1) = 2 + 2(1) = 4 which is an even number so the base case holds \\
IH: For any number k of square blocks you lie down, the number of sides are even which we can represent as f(k) = 2k+2 = even integers\\
Prove f(k+1) so if you add one more block then f(k+1) = 2(k+1) + 2 \\
= 2k+2 +2 or f(k) + 2 and if we are assuming f(k) is even then because of what we know about even integers we know and even int plus 2 is still even. \\
So f(k+1) holds for all numbers of blocks. 

% #3
\item {\bf Question 3} \\
a) \\
Base Case: when n = 14 then the P(n) = 8c + 3c + 3c = 14c \\
so the base case holds \\
IH: Assume for any P(k) where k is greater than or equal to 14 k can be formed by using just 8c and 3c pieces. \\
To prove P(k+1) then we have two cases \\
Case 1: Suppose k is formed with at least two 3c piece and one 8 piece used to make up k cents. \\
Then k+1  we have to substitute 8c with 3 3c pieces . 
Case 2: Suppose k is formed with at least one 8c piece and two 3c pieces \\
then k+1 must substitute 8c with 3 3c pieces .\\
so k+1 holds as long as we have five 3 cent pieces. \\
b) \\
Base Cases: \\
	14c = 8c + 3c + 3c \\
	15c = 3c + 3c + 3c + 3c + 3c  \\
	16c = 8c + 8c  \\
	17c = 8c + 3c + 3c \\
	18c = 3c + 3c + 3c + 3c + 3c + 3c \\
	19c = 8c + 8c + 3c \\
	20c = 8c + 3c + 3c + 3c \\
	21c = 3c + 3c + 3c + 3c + 3c + 3c + 3c \\
	let k be greater or equal to 21  \\
IH:
	Assume that for all y that 14 less than or equal to y and less than or equal to k. for postage of y cents. \\
	All postage of y cents can be formed using 8c and 3c pieces. \\
	Since k greater than or equal to 21 then we have postage for k - 7 is greater than or equal to 14 \\
	So, to prove k+1 we just add 8c to k - 7 postage \\
	So k + 1 holds when we have at least 2 8c pieces and 2 3c pieces. \\ 


% #4
\item {\bf Question 4} \\
Base Case: We know p(n) is true for an infinite amount of n's so we pick P(n) where it holds. \\
IH: Assume P(k+1) holds for any positive integer k.\\
Prove: That the function holds for all values less than P(k+1). Since we know that P(k+1) implies P(k) then we know that P(k) holds. Which means that we know P(n) holds for any positive integer. 


\end{enumerate}
\end{document}
