% some document settings
\documentclass[12pt]{article}
\setlength{\oddsidemargin}{12pt}
\setlength{\textwidth}{6.5in}
\setlength{\textheight}{9in}
\pagestyle{empty}
\setlength{\parskip}{7pt plus 2pt minus 2pt}

% include some needed libraries
\usepackage{amssymb}
\usepackage{amsmath}
\usepackage{amsthm}
\usepackage{enumitem}
\let\biconditional\leftrightarrow

\begin{document}
% begin editing

% title part
\title{
\textbf{COMS 230: Discrete Computational Structures}\\
Homework \# 8 \\}
\author{Larisa Andrews}
\date{\today}
\maketitle

\begin{enumerate}

% #1
\item {\bf Question 1} \\
a)\\ 
	To prove S is a subset of A we must prove all elements in the set S\\
	are multiples of 7. \\	
	Base Case: P(n) = 7n $\in S$, P(1) = 7(1) = 7 \\
	so the base case holds \\
	Ind: P(k) = 7k $ \in S$\\
	Prove that P(k+1) $\in S$ \\
	P(k+1) = 7(k+1) = 7k + 7 \\
	From the base case we know 7 $\in S$ so it follows by the definition\\   it follows that P(k+1) $\in S$ \\
	Therefore S contains the multiples of 7 so it is a subset of A.\\
b)\\
	Base Case: 3 $\in S$ and 3(1) = 3 so $3 \in A$ \\
	Ind: Assume if s $\in S$ and t $\in S => s+t \in S$ \\
	Prove that for element in A s + t is a multiple of 3. \\
	We can show this by saying that for every s and t $\in A$ we know s is divisible by 3 and t is divisible by 3. \\
	Then it follows that s + t would also be divisible by 3. \\
	So A is a subset of S\\
	
	
	

% #2
\item {\bf Question 2} \\
First Step:\\
	Case 1: k,m,n are all the same so when k = 0, 1 $\in S$\\
	Case 2: k,m,n are all different so when k = 0, m = 1, n = 2, \\
	then $75 \in S$ \\
	Case 3: for k,m,n two of the variables are the same so when k = 0, \\
	m = 1, then $5 \in S$ or $15 \in S$ or $3 \in S$ \\
Ind Step:\\
	Case 1: we would add $30^k$ to the set \\
	Case 2: we would add $2^k3^m5^n$ to the set \\
	Case 3: we would add $6^k5^m$ or $2^k15^m$ or $10^k3^m$ \\

% #3
\item {\bf Question 3} \\
a) \\
Base Case: (0,0) $\in S$ and 0+0 mod 4 = 0\\
so the base case holds \\
Ind: Assume that if (a,b) $\in S$ then a + b mod 4 = 0 because we have only added the base case to the set. \\
Prove for all steps that a+b mod 4 = 0 holds \\
Case 1: (a,b+4) $=> a+b+4 =>$ from the induction set we assume a+b mod 4 = 0 and since 4 mod 4 = 0 then a+b+4 mod 4 = 0\\
Case 2:  (a+1, b+3) $=> (a+b)+4 =>$ from case 1 we know it holds\\
Case 3: (a+2, b+2) $=> (a+b)+4 =>$ from case 1 we know it holds\\
b) \\
Disprove by example:
(3,1), 3 + 1 = 4/4 = 0 but is not in the set.\\ 
(3,5), 3 + 5 = 8/4 = 2 but is not in the set.\\
Modified S = $\{(a,b)|a,b \in \mathbb{N}, (a+b) mod 4 = 0\}$




% #4
\item {\bf Question 4} \\
n(T) base: T = 1 then n(1) = 1 \\
Ind: n(T) = 1 + n(T1) + n(T2) \\
where T1 is the number of vertices of the right sub-tree and T2 is for the left sub-tree \\
l(T) base: T = 1(meaning a single vertex) then l(T) = 1\\
Ind: l(T) = l(T1) + l(T2) \\
Base Case: n(T) = 1 if T = 1 \\
so n(1) = 2l(1) - 1 = 2-1 = 1 so the base case holds \\
Ind: Assume T1 and T2 are FBTs that are left and right subtrees of T \\
so n(T) = 1 + n(T1) + n(T2) \\
prove that n(T) is equal to 2l(T) + 1 \\
$=>$ n(T1) - 2l(T1) -1 and n(T2) - 2l(T2) - 1 \\
$=>$ n(T) = 2l(T1) - 1 + 2l(T2) - 1 + 1 \\
$=>$ 2l(T1) + 2l(T2) + 1 \\
$=>$ 2(l(T1) + l(T2)) + 1 \\
since l(T) = l(T1) + l(T2) then \\
n(T) = 2(T) + 1 \\

% #5
\item {\bf Question 5} \\
a) \\
Base Case: (0,0) $\in L$ because 0-0 = 0 mod 4 = 0 so it  holds \\
Ind: assume (a,b) $\in L$ then (a+4,b) $\in L$ and (a,b+4) $\in L$ and (a+2, b+2) $\in L$ \\
b)\\
Base Case: P(0,0) = 0-0 mod 4 = 0 $\in L$, (0,0) $\in L'$ so it holds \\
Ind: Assume P(a,b) = a-b mod 4 = 0 according to L's definition \\
Prove P(a+4, b), P(a,b+4), P(a+2,a+2) hold \\
P(a+4, b) = (a+4-b) $=>$ (a-b) + 4 which follows by IH that it holds \\
P(a,b+4) = (a,b+4) $=>$ a-b+4 which follows by the IH that it holds \\
P(a+2,a+2) = (a+2, a+2) $=>$ (a-b)+4 which follows by the IH that it holds\\
Since for every step in L' holds for L then L is a subset of L' \\
 
c) \\
Base Case: (0,0) $\in L'$ and 0-0 = 0 mod 4 = 0, so $\in L$ \\
ind: Assume that if a $\in L'$ and b $\in L'$ then a -b mod 4 = 0\\
Prove that all elements in L' mod 4 = 0\\
Case 1: (a+4,b) $\in L'$ and if a-b mod 4 = 0 and 4 mod 4 = 0 then it follows that a+4 - b mod 4 is also zero. When you know that a will be multiples of 4 or 2 if b is 2. \\
Case 2: (a, b+4) $\in L's$ this case holds because it will result in the negative version of case 1 \\
Case 3: (a+2, b+2) $in L's$ since a-b mod 4 = 0 then (a-b)+4 mod 4 will also be mod 0. \\
So since the elements in L' mod 4 = 0 then L is a subset of L'\\




\end{enumerate}
\end{document}
